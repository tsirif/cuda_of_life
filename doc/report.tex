%&preamble
% Save static part as preamble.tex and use command:
% xelatex -ini -shell-escape -job-name="preamble" "&xelatex preamble.tex\dump"
% to produce preamble.fmt

% Only for xelatex and lualatex. It provides an automatic and unified interface to feature-rich AAT and OpenType fonts.
% https://ctan.org/pkg/fontspec
\usepackage{fontspec}
% fix bug with fontspec: http://tex.stackexchange.com/a/102610/78791
\ExplSyntaxOn
\let\tl_length:n\tl_count:n
\ExplSyntaxOff

\setmainfont{DejaVu Serif}
\renewcommand{\contentsname}{Περιεχόμενα}
\renewcommand{\listfigurename}{Λίστα Σχημάτων}
\renewcommand{\figurename}{Σχήμα}
\renewcommand{\lstlistingname}{Καταχώρηση}
\renewcommand{\lstlistlistingname}{List of \lstlistingname s}

\date{05/10/2015}
\title{3η Εργασία στα Παράλληλα και Διανεμημένα Συστήματα\\
Game of Life}
\author{Χρήστος Τσιριγώτης, 7792\\
Ορέστης Φλώρος-Μαλιβίτσης, 7796\\
  Τομέας Ηλεκτρονικής,\\
  Τμήμα Ηλ. Μηχανικών / Μηχανικών ΗΥ,\\
  Αριστοτέλειο Πανεπιστήμιο Θεσσαλονίκης}
\begin{document}
\maketitle
\tableofcontents
\listoffigures
\newpage

\chapter*{Δομή του Project}

\begin{description}
	\item[bboard*/] Φάκελοι με τις διάφορες εκδοχές σε bitfields.
	\item[bin/] Φάκελος με τα binaries των διάφορων εκδόσεων. Το build έγινε στον diades.
	\item[doc/] Φάκελος με τα \LaTeX{} αρχεία για την παραγωγή της αναφοράς.
	\item[single/] Φάκελος με τις εκδοχές σε cpu, CUDA 1 cell per thread.
	\item[game-of-life.pdf] Η εκφώνηση.
	\item[report.pdf] Αυτή η αναφορά.
\end{description}

\noindent Για build τρέχουμε:
\begin{lstlisting}[style=Bash]
$ mkdir -p bin
$ cd bin
$ cmake ..
$ make
\end{lstlisting}
\chapter{Εισαγωγή}

Στην εργασία αυτή μας ζητείται η υλοποίηση του
\href{https://en.wikipedia.org/wiki/Conway%27s_Game_of_Life}{Conway's Game of Life}
σε \href{https://en.wikipedia.org/wiki/CUDA}{CUDA}.
Οι κανόνες του Game of life (GOL) είναι οι εξής:

Σε έναν τετραγωνικό πίνακα $N \times N$ κάθε κελί μπορεί να βρίσκεται σε δύο καταστάσεις: 
ζωντανό(1) ή νεκρό(0).
Σε κάθε γενεά του παιχνιδιού τα κελιά ανανεώνονται σύμφωνα με 4 κανόνες: 
\begin{enumerate}
\item Κάθε ζωντανό κελί με λιγότερους από δύο ζωντανούς γείτονες, πεθαίνει.
\item Κάθε ζωντανό κελί με δύο ή τρεις ζωντανούς γείτονες επιβιώνει στην επόμενη γενεά.
\item Κάθε ζωντανό κελί με περισσότερους από τρεις ζωντανούς γείτονες πεθαίνει.
\item Κάθε νεκρό κελί με ακριβώς τρεις ζωντανούς γείτονες ζωντανεύει.
\end{enumerate}

Ζήτημα μας είναι η υλοποίηση αλγορίθμου που ανανεώνει τον πίνακα του παιχνιδιού για κάποιον αριθμό επαναλήψεων.
\chapter{Υλοποίηση σε CUDA}

Υλοποιήθηκαν οι εξής εκδόσεις του προγράμματος:

\begin{enumerate}
\item Σειριακός κώδικας (single/cpu1.c)
\item Κώδικας σε Openmp μαζί με διάφορα micro-optimizations (single/omp.c).
\item Cuda ένα κελί ανά νήμα (single/cuda-single.cu).
\item Cuda Πολλαπλά κελιά ανά νήμα: υλοποίηση σε bitboard 32-bit (bboard2d).
\item Cuda Πολλαπλά κελιά ανά νήμα: υλοποίηση σε bitboard 32-bit με shared memory (bboard2d-shared).
\item Cuda Πολλαπλά κελιά ανά νήμα: υλοποίηση σε bitfields 64-bit (bboard1x64).
\item Cuda Πολλαπλά κελιά ανά νήμα: υλοποίηση σε bitfields 64-bit με shared memory (bboard1x64-shared).
\end{enumerate}

Η υλοποίηση της σειριακής, openmp και ένα κελί ανά νήμα υλοποίησης είναι αρκετά απλή και δεν
περιγράφεται. Οι 4 τελευταίες εκδόσεις λειτουργούν με παρόμοιο τρόπο αλλά
η ανάπτυξή τους διαχωρίστηκε από ένα σημείο και μετά
και για αυτό αντιμετωπίζονται σαν ξεχωριστά project
(διαφορετικό CMakelists.txt, διαφορετικός φάκελος κτλπ).

Τα bitboards επιλέχτηκαν καθώς ελαττώνουν την χρήση της global μνήμης σύμφωνα με τον μέγεθός τους
και επιτρέπουν την αποφυγή for και if block μέσω της χρήσης bitwise πράξεων.

\begin{sloppypar}
Η μετατροπή από κανονικό σε "tiled" πίνακα γίνεται στα αρχεία converters.cu
με την βοήθεια των συναρτήσεων \lstinline!convert_from_tiled()!
και \lstinline!convert_to_tiled()!.
Ο τύπος των δεδομένων σε tiled μορφή είναι \lstinline!bboard!
και στα αρχεία utils.h ορίζονται διάφορα macro για την προσπέλαση των bits.
Η πρόσβαση στην global μνήμη της συσκευής γίνεται μέσω της χρήσης της \lstinline!cudaMallocPitch()!
που θεωρείται καλύτερη για την πρόσβαση σε δισδιάστατους πίνακες.
Στην global μνήμη αποθηκεύονται δύο πίνακες,
\lstinline!bboard* d_board! και
\lstinline!bboard* d_help!.
Ο πρώτος κρατάει το αποτέλεσμα της τρέχουσας γενιάς
και ο δεύτερος χρησιμοποιείται για τον υπολογισμό της επόμενης.
\end{sloppypar}

\begin{sloppypar}
Ο υπολογισμός της κάθε γενιάς γίνεται στα αρχεία calculator.cu μέσω της \lstinline!calculate_next_generation()!.
\end{sloppypar}

\section{bboard2d}
\begin{sloppypar}
Στον πίνακα \lstinline!bboard neighbors[3][3]! αποθηκεύονται τα γειτονικά tiles του κάθε thread.
Έτσι, στην θέση [1][1] είναι τα τοπικό tile,
στην θέση [0][0] το πάνω αριστερά,
στην θέση [2][1] το κάτω κ.ο.κ.
\end{sloppypar}
\begin{sloppypar}
Στην έκδοση χωρίς χρήση shared memory το κάθε thread γεμίζει τα 9 κελιά του \lstinline!neighbors! με απευθείας πρόσβαση στην global μνήμη.
Στην έκδοση με shared memory, το κάθε thread γεμίζει μια θέση που του αναλογεί από τον πίνακα
\lstinline!__shared__ bboard tiles[]! και
τα threads που βρίσκονται στα σύνορα του block αναλαμβάνουν να γεμίσουν τις θέσεις που
αντιστοιχούν στα tiles με τα οποία συνορεύει το block.
Τέλος, μετά την \lstinline!__syncthreads();! οι θέσεις του \lstinline!neighbors! γεμίζονται
μέσω της πρόσβασης στον \lstinline!tiles!.
\end{sloppypar}

\begin{sloppypar}
Εδώ, υπάρχει μια διαφοροποίηση ανάλογα με το αν η διάσταση $N$ του πίνακα διαιρείται τέλεια με το 8.
Όταν αυτό ισχύει χρησιμοποιείται η συνάρτηση \lstinline!calculate_next_generation_no_rem()!
από την \lstinline!main()! αλλιώς η \lstinline!calculate_next_generation()!.
Αυτό συμβαίνει γιατί η μέθοδος με τις bitwise πράξεις της πρώτης είναι δυσκολότερο να εφαρμοστεί και στην δεύτερη.
\end{sloppypar}

Στην \lstinline!calculate_next_generation()! ο υπολογισμός γίνεται με απλή for στις διαστάσεις
\lstinline!WIDTH! και \lstinline!HEIGHT! του bitboard.
Εφαρμόζονται μερικές μικρο-βελτιώσεις με την χρήση του preprocessor ώστε να μεταφερθούν οι
οριακές περιπτώσεις εκτός της for και να αποφευχθούν μερικές if.

Στην \lstinline!calculate_next_generation_no_rem()! αρχικά εφαρμόζονται μάσκες και bitwise
πράξει στα στοιχεία του πίνακα \lstinline!neighbors! ώστε να ενωθούν σε ένα μεγαλύτερο bitboard τύπου \lstinline!ext_bboard!. 
Για παράδειγμα αν έχουμε τον εξής πίνακα \lstinline!neighbors!:\\
\begin{tabular}{ | c | c | c | c | c | c | c | c |}
\hline
1 & 1 & 1 & 0 & 0 & 1 & 0 & 1 \\ \hline
1 & 0 & 1 & 0 & 1 & 1 & 0 & 1 \\ \hline
1 & 0 & 1 & 0 & 0 & 0 & 1 & 0 \\ \hline
1 & 1 & 1 & 0 & 1 & 1 & 1 & \cellcolor{blue!25}0 \\ \hline
\end{tabular} \
\begin{tabular}{ | c | c | c | c | c | c | c | c |}
\hline
1 & 1 & 1 & 1 & 0 & 0 & 1 & 0 \\ \hline
1 & 1 & 1 & 0 & 0 & 0 & 1 & 0 \\ \hline
1 & 0 & 1 & 0 & 1 & 0 & 1 & 0 \\ \hline
\cellcolor{blue!25}1 & \cellcolor{blue!25}0 & \cellcolor{blue!25}0 & \cellcolor{blue!25}1 & \cellcolor{blue!25}0 & \cellcolor{blue!25}0 & \cellcolor{blue!25}1 & \cellcolor{blue!25}0 \\ \hline
\end{tabular} \
\begin{tabular}{ | c | c | c | c | c | c | c | c |}
\hline
0 & 0 & 1 & 1 & 1 & 0 & 0 & 0 \\ \hline
1 & 1 & 0 & 1 & 0 & 0 & 0 & 1 \\ \hline
0 & 1 & 0 & 0 & 1 & 0 & 1 & 1 \\ \hline
\cellcolor{blue!25}1 & 0 & 1 & 0 & 1 & 0 & 0 & 1 \\ \hline
\end{tabular} \\
\begin{tabular}{ | c | c | c | c | c | c | c | c |}
\hline
1 & 0 & 1 & 0 & 0 & 0 & 0 & \cellcolor{blue!25}1 \\ \hline
0 & 0 & 0 & 0 & 0 & 1 & 1 & \cellcolor{blue!25}1 \\ \hline
0 & 1 & 1 & 0 & 1 & 1 & 1 & \cellcolor{blue!25}1 \\ \hline
1 & 1 & 1 & 0 & 0 & 1 & 1 & \cellcolor{blue!25}0 \\ \hline
\end{tabular} \
\begin{tabular}{ | c | c | c | c | c | c | c | c |}
\hline
1 & 1 & 0 & 1 & 0 & 1 & 0 & 1 \\ \hline
1 & 0 & 0 & 0 & 1 & 1 & 1 & 0 \\ \hline
0 & 1 & 1 & 1 & 1 & 0 & 1 & 0 \\ \hline
1 & 0 & 1 & 0 & 1 & 1 & 0 & 1 \\ \hline
\end{tabular} \
\begin{tabular}{ | c | c | c | c | c | c | c | c |}
\hline
\cellcolor{blue!25}1 & 0 & 0 & 1 & 1 & 0 & 1 & 1 \\ \hline
\cellcolor{blue!25}0 & 0 & 0 & 0 & 0 & 0 & 0 & 1 \\ \hline
\cellcolor{blue!25}0 & 1 & 1 & 1 & 0 & 0 & 1 & 0 \\ \hline
\cellcolor{blue!25}0 & 0 & 0 & 1 & 1 & 1 & 1 & 0 \\ \hline
\end{tabular} \\
\begin{tabular}{ | c | c | c | c | c | c | c | c |}
\hline
0 & 1 & 1 & 1 & 1 & 1 & 1 & \cellcolor{blue!25}1 \\ \hline
1 & 1 & 1 & 1 & 1 & 0 & 1 & 1 \\ \hline
1 & 1 & 0 & 1 & 1 & 0 & 1 & 0 \\ \hline
1 & 0 & 0 & 0 & 0 & 0 & 0 & 1 \\ \hline
\end{tabular} \
\begin{tabular}{ | c | c | c | c | c | c | c | c |}
\hline
\cellcolor{blue!25}1 & \cellcolor{blue!25}0 & \cellcolor{blue!25}1 & \cellcolor{blue!25}0 & \cellcolor{blue!25}1 & \cellcolor{blue!25}1 & \cellcolor{blue!25}1 & \cellcolor{blue!25}1 \\ \hline
1 & 0 & 0 & 0 & 0 & 0 & 0 & 1 \\ \hline
1 & 1 & 0 & 0 & 1 & 1 & 0 & 1 \\ \hline
1 & 0 & 1 & 0 & 1 & 0 & 0 & 1 \\ \hline
\end{tabular} \
\begin{tabular}{ | c | c | c | c | c | c | c | c |}
\hline
\cellcolor{blue!25}1 & 1 & 1 & 0 & 1 & 1 & 1 & 0 \\ \hline
1 & 0 & 0 & 0 & 1 & 0 & 1 & 1 \\ \hline
1 & 1 & 0 & 1 & 0 & 1 & 1 & 1 \\ \hline
0 & 0 & 0 & 0 & 1 & 1 & 0 & 0 \\ \hline
\end{tabular}
\begin{center}
	Θα πρέπει αν μεταφερθεί ως εξής:\\
	\begin{tabular}{ | c | c | c | c | c | c | c | c | c | c |}
	\hline
	\cellcolor{blue!25}0 & \cellcolor{blue!25}1 & \cellcolor{blue!25}1 & \cellcolor{blue!25}1 & \cellcolor{blue!25}0 & \cellcolor{blue!25}1 & \cellcolor{blue!25}1 & \cellcolor{blue!25}1 & \cellcolor{blue!25}0 & \cellcolor{blue!25}1 \\ \hline
	\cellcolor{blue!25}1 & 1 & 1 & 0 & 1 & 0 & 1 & 0 & 1 & \cellcolor{blue!25}1 \\ \hline
	\cellcolor{blue!25}1 & 1 & 0 & 0 & 0 & 1 & 1 & 1 & 0 & \cellcolor{blue!25}0\\ \hline
	\cellcolor{blue!25}1 & 0 & 1 & 1 & 1 & 1 & 0 & 1 & 0 & \cellcolor{blue!25}0\\ \hline
	\cellcolor{blue!25}0 & 1 & 0 & 1 & 0 & 1 & 1 & 0 & 1 & \cellcolor{blue!25}0\\ \hline
	\cellcolor{blue!25}1 & \cellcolor{blue!25}1 & \cellcolor{blue!25}0 & \cellcolor{blue!25}1 & \cellcolor{blue!25}0 & \cellcolor{blue!25}1 & \cellcolor{blue!25}1 & \cellcolor{blue!25}1 & \cellcolor{blue!25}1 & \cellcolor{blue!25}1\\ \hline
	\end{tabular} \
\end{center}

Αυτό γίνεται μέσω της \lstinline!bboard_to_ext()!:
\begin{lstlisting}[caption={Extend bboard}, escapechar=$, breaklines=true]
__device__
bboard ext_to_bboard(ext_bboard val) {
    bboard res = 0;
    for (int i = 1; i < EXT_HEIGHT - 1; i++) {
        for (int j = 1; j < EXT_WIDTH - 1; j++) {
            if (EXT_BOARD_IS_SET(val, i, j)) SET_BOARD(res, i - 1, j - 1);
        }
    }
    return res;
}

...

ext_bboard res = 0;
for (int i = 0; i < 3; i++) {
    for (int j = 0; j < 3; j++) {
        res |= bboard_to_ext(neighbors[i][j], i, j);
    }
}
\end{lstlisting}

Ο τελικός υπολογισμός γίνεται μέσω της \lstinline!gol()!.
Οι πράξεις που γίνονται στην συνάρτηση βρέθηκαν \href{http://www.onjava.com/pub/a/onjava/2005/02/02/bitsets.html?page=2}{εδώ}.
Ουσιαστικά, υπολογίζονται τα bits που έχουν άθροισμα ζωντανών γειτόνων 3 και 2.

\section{bboard1x64}

Λειτουργεί με παρόμοιο τρόπο με διαφορές στον πίνακα \lstinline!neighbors! και τις bitwise πράξεις που εκτελούνται.
Εδώ, δεν διαχωρίζεται η συνάρτηση ανάλογα με το αν διαιρείται το $N$ με το 8.

\chapter{Αποτελέσματα}

Όλες οι μετρήσεις έγιναν στον diades για τα αρχικά πλέγματα.
Ο έλεγχος ορθότητας των υλοποιήσεων έγινε συγκρίνοντας το md5sum των αποτελεσμάτων τους.
Για κάθε αρχικό πλέγμα .bin και για όλους τους αριθμούς επαναλήψεων και οι 7 εκδόσεις παρήγαγαν ίδιο .bin αρχείο σαν αποτέλεσμα.

Από τις μετρήσεις στο \hyperref[fig:speedup]{\figurename{} \ref{fig:speedup}}
παρατηρούμε ότι οι υλοποιήσεις σε CUDA είναι μέχρι και 900 φορές γρηγορότερες σε σχέση με τον σειριακό κώδικα.
Επίσης, φαίνεται πως η bboard είναι πιο αργή για $N$ 500 (δεν διαιρείται με το 8) από την bboard1x64 αλλά στην συνέχεια είναι καλύτερη για $N$ 1000, 2000 και 4000.
Καλύτερα αποτελέσματα συνολικά έχει η bboard1x64-shared.

Στο \hyperref[fig:cuda-N]{\figurename{} \ref{fig:cuda-N}} φαίνονται τα μεγάλα οφέλη των bitfields.
Στο \hyperref[fig:cuda-iter]{\figurename{} \ref{fig:cuda-iter}} φαίνεται ότι ο χρόνος εκτέλεσης αυξάνεται γραμμικά με τον αριθμό επαναλήψεων.

\begin{figure}[h]
\centering
\subfloat[]{\label{fig:speedup:a}\includegraphics[height=0.35\textheight]{plots/speedup_iters_10.pdf}}\
\subfloat[]{\label{fig:speedup:b}\includegraphics[height=0.35\textheight]{plots/speedup_iters_100.pdf}}\
\subfloat[]{\label{fig:speedup:c}\includegraphics[height=0.35\textheight]{plots/speedup_iters_1000.pdf}}\
\caption{Επιτάχυνση σε σχέση με σειριακό κώδικα cpu.}
\label{fig:speedup}
\end{figure}

\begin{figure}[h]
\centering
\subfloat[]{\label{fig:cuda-N:a}\includegraphics[height=0.35\textheight]{plots/abs_cuda_10.pdf}}\
\subfloat[]{\label{fig:cuda-N:b}\includegraphics[height=0.35\textheight]{plots/abs_cuda_100.pdf}}\
\subfloat[]{\label{fig:cuda-N:c}\includegraphics[height=0.35\textheight]{plots/abs_cuda_1000.pdf}}\
\caption{Σύγκριση των υλοποιήσεων σε CUDA}
\label{fig:cuda-N}
\end{figure}


\begin{figure}[h]
\centering
\subfloat[]{\label{fig:cuda-iter:a}\includegraphics[width=0.5\linewidth]{plots/abs_cuda_N_500.pdf}}
\subfloat[]{\label{fig:cuda-iter:b}\includegraphics[width=0.5\linewidth]{plots/abs_cuda_N_1000.pdf}}\
\subfloat[]{\label{fig:cuda-iter:c}\includegraphics[width=0.5\linewidth]{plots/abs_cuda_N_2000.pdf}}
\subfloat[]{\label{fig:cuda-iter:d}\includegraphics[width=0.5\linewidth]{plots/abs_cuda_N_4000.pdf}}\
\caption{Σύγκριση των υλοποιήσεων σε CUDA}
\label{fig:cuda-iter}
\end{figure}

\end{document}
